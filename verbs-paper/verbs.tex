\documentclass[11pt]{article}
\usepackage{freerbmt09}
\usepackage{ucs}
\usepackage[utf8x]{inputenc}
\usepackage[T1]{fontenc}
\usepackage{times}
\usepackage{natbib}
\usepackage{graphicx}
\usepackage{rotating}
\usepackage{url}
\usepackage[small,bf]{caption}
\usepackage{alltt}
\usepackage{linguex}
\usepackage{latexsym}

\usepackage{hyperref}
%\author{NN\\  X \\ X \\  X \\  {\tt \small   X@X} \And  NN \\  X \\  X \\  X \\    {\tt \small  X@X}}
\author{Jimmy O'Regan\\  Eolaistriu Technologies \\  Tipperary, Ireland \\  {\tt \small   joregan@gmail.com} }


\title{Aspect Selection of Slavic Verbs in an English--Polish Translator Built with the Apertium Platform}

\begin{document}

\maketitle

  \begin{abstract}
   Slavic languages\footnote{
   When we refer in general terms to ``Slavic languages'', we specifically exclude 
   the Bulgarian and Macedonian languages, which diverge from 
   the rest of the Slavic family in a number of significant ways;
   however, the question of aspect covered in this paper is not
   one of them.} 
   pose a number of problems in translation to non-Slavic
   languages. We describe a solution to the problem of Slavic aspect pairs
   suitable for use within a shallow-transfer machine translator
   built on the Apertium platform, with particular focus on the
   Polish--English language pair. We consider the problems posed
   by {\emph Aktionsarten} and other exceptions to the notion of
   aspect ``pairs''.
  \end{abstract}

\section{Introduction}
Slavic verbs are generally considered to have ``aspect pairs'',
consisting of an {\it imperfective} and {\it perfective} verb: 
(generally) the former signifies that the action is on-going; 
the latter, completed. Table~\ref{table:vbpairs} contains
some examples of Polish verb pairs.

\begin{table}[htdp]
\caption{Imperfective \& Perfective Verb Pairs}
\label{table:vbpairs}
\begin{center}
\begin{tabular}{|l|l|l|}
\hline
{\sc Imperf.} & {\sc Perf.} & English \\
\hline
{\it je{\'s\'c}} & {\it zje{\'s\'c}} & to eat \\
{\it widzie{\'c}} & {\it zobaczy{\'c}} & to see \\
{\it uczy{\'c}} & {\it nauczy{\'c}} & to teach \\
\hline
\end{tabular}
\end{center}
\end{table}

As can be seen in the table, many of these verb pairs contain
a perfective verb with a preposition as a prefix: unlike other
uses of prepositional prefixes, these do not change the meaning
of the base verb: they serve only to provide a difference 
between imperfective and perfective and are usually referred to as
``empty'' prefixes.

In reality, the situation is much more complicated: verbs of motion can
have triples: imperfective indeterminate ({\it ``destination unknown''}), 
imperfective determinate ({\it ``destination known, action incomplete''}),
and perfective ({\it ``action completed''}); some verbs are {\it dual aspect}:
they can be used both as perfective and imperfective verbs; many common
verbs have a {\it frequentative} version; and, finally, recently\footnote{
{\it ``In Slavic aspectology, the distinction between aspect and Aktionsart
began [to be] accepted in the early 1930s. However, the notion of Aktionsart did not make
it into textbooks on Slavic grammars until the second half of the twentieth century,
and it did not make it into Slavic dictionaries until very recently.''}
\citep[Ch.~2]{mlyn}}
the distinction between aspect and {\it Aktionsarten} has become more common.\\

Polish verbs have been classified by Aktionsarten~\citep[Ch.~4]{mlyn} into
five classes, as in Table~\ref{table:aktionsarten}. 

\begin{table}[htdp]
\caption{Classification of Polish verbs by Aktionsarten}
\label{table:aktionsarten}
\begin{center}
\begin{tabular}{|l|c|c|c|c|}
\hline
& ep & {\it po-} & {\it -n\k{a}-} & mpc\\
\hline
\hline
Class$_{1}$ & yes & & & \\
\hline
Class$_{2}$ &  & yes & & \\
\hline
Class$_{3}$ & yes & yes & & \\
\hline
Class$_{4}$ & yes & yes & yes & \\
\hline
Class$_{5}$ & & & & yes \\
\hline
\end{tabular}
\end{center}
\end{table}

In the table, {\it ep} refers to ``empty prefix'', {\it po-} refers
to the prepositional ``po'', whether as an empty prefix, or as the
deliminative prefix, {\it -n\k{a}-} refers to the semelfactive infix,
and {\it mpc} refers to ``morpho-phonological change''. For our
purposes, Classes 1, 2, and 5 are equivalent: these refer to aspect
pairs. The remaining classes 3 and 4 are more interesting: they 
refer to groups of 3 and 4 verbs, respectively.

In these Classes, {\it po-} is the deliminative prefix: it represents
a perfective aspect with an other-than-usual outcome. In the example
of the Class 3 group containing the verb {\it czyta\'{c}}, ``to read'',
{\it przeczyta\'{c}} is the usual perfective, meaning ``to read'', but
with the connotation that the reading was completed; {\it poczyta\'{c}},
on the other hand, means ``to read for a while'': to read a book from
cover to cover is the former; to read a few pages, throw it aside, and 
vow never to touch it again is the latter\footnote{Not, I hope, a 
self-fulfilling prophecy}.

Class 4 are unitisable processes~\citep{aalstein07}: {\it puka\'{c}}
``to knock'' yields the perfectives {\it pukn\k{a}\'{c}} ``to knock once'', 
{\it popuka\'{c}} ``to knock for a while'', and the ``empty'', regular 
{\it zapuka\'{c}}.\\

Apertium~\citep{corbi05oss} is a shallow-transfer Machine Translation platform,
originally designed for the Romance languages of Spain. Later development has
extended the system to enable the development of translators between more
divergent language pairs, such as Basque--Spanish~\citep{ginesti09}.

Apertium is constructed in a modular, assembly-line fashion. XML dictionaries
are compiled to finite-state transducers; these transducers are reversible, so
the same monolingual dictionaries can be used both for analysis and generation,
and the same bilingual dictionary can be used in both translation directions.

After morphological analysis, the analysed text stream is disambiguated using
a bigram tagger based on first order hidden Markov models (HMMs), before
being passed to either 1 (for very closely related languages) or 3 
(less related) layers of structural transfer. In the 3-layer system,
bilingual translation is handled by the first transfer layer, with the
other two layers handling inter-chunk operations (such as gender 
agreement), and chunk post-processing respectively.\\

Work began on a Polish--English language pair almost two years ago,
but when various limitations in the platform prevented the completion of
the system, the developers lost interest and pursued other projects.

Polish, like the other Slavic languages, is a highly
inflected language. Although word order is typically SVO, Polish can almost
be considered to be a free word order language, with inflected endings
{\it usually} providing the distinction between subject and object. This 
alone does not pose a problem for structural transfer, though it does
mean that six times the usual number of inter-chunk rules are theoretically
required: in practice, however, not all possible variations are typically 
necessary. 

In addition, cases often serve to in place of prepositions; the genitive
case in particular poses several difficulties, as does the ambiguous
preposition ``z'', which when followed by a noun phrase in the genitive
case generally means ``from''; by the instrumental, ``with''. We tried
to handle this by annotating ``z'' with both expected cases, and
allowing the tagger to choose the correct one, which works well for
many cases: {\it z Francji} -- ``from France''; {\it z mlekiem} --
``with milk'', but fails in more complicated cases:

\begin{center}
\begin{table}[htdp]
z$_{\it Pr.Ins}$ brata$_{\it N.Gen}$ {\.{z}on\k{a}}$_{\it N.Ins}$ \\
with (my) brother's wife \\
z$_{\it Pr.Gen}$ miasta$_{\it N.Gen}$ samochodem$_{\it N.Ins}$\\
from town by car
\end{table}
\end{center}

In the English--Polish direction, the biggest problem is that the
apertium transfer component can only use a single translation per lemma;
for Polish verbs, in the Polish--English direction, this poses no problem,
as both perfective and imperfective verb forms can be matched to the
same translation, but in the English--Polish direction, we were unable to
translate many types of verb phrase properly.

\begin{figure}[pt]
\begin{center}
\includegraphics*[width=7.5cm]{aspectselect.png}
\end{center}
\caption{Diagram of the English--Polish pipeline, with the aspect selection
module highlighted in black.}
\label{figure:modules}
\end{figure}

\section{Aspect selection}

One of several solutions we attempted was to use the single-level transfer
mode with a custom dictionary consisting only of verbs, but this proved both
unwieldy, as well as introducing new problems, such as missing word marks for
non-verbs.

The final solution came during the development of the apertium-cy~\citep{tyers2009acd}
system. To facilitate the post-processing of Welsh to introduce initial mutation, the
apertium-transfer component was modified to operate in a monolingual, dictionary-less
mode. This provided us with a new opportunity to test aspect selection.

\begin{figure}[p]
\begin{small}
\begin{alltt}
<rule>
  <pattern>
    <pattern-item n="is/was"/>
    <pattern-item n="gerund"/>
  </pattern>
  <action>
    <out>
      <lu>
        <get-case-from pos="1">
          <clip pos="2" side="sl" 
           part="lemh"/>
        </get-case-from>
        <lit v="imperf"/>
        <clip pos="1" side="sl" 
         part="temps"/>
        <lit v="PD"/>
        <lit v="ND"/>
        <clip pos="2" side="sl" 
         part="lemq"/>
      </lu>  
    </out>
  </action>
</rule>
\end{alltt}
\end{small}
\caption{A sample rule that matches the verb ``be'' followed by a
gerund. In this rule, the tense (``temps'') is taken from ``be'', which
is discarded; the primary verb is tagged as imperfective, and marked so
that person and number are to be added in transfer (in the real rule, 
where possible, the person information present in the form of ``be'' is
used; that processing is omitted here for sake of demonstration).}
\label{figure:rule}
\end{figure}

Instead of the 3 level transfer we had been using, we experimented with a 4 level
system, where the English source was pretranslated to pseudo-English, where verbs
were marked for aspect (Figure~\ref{figure:modules}). To reduce the number of 
rules required in the previous first level, where translation is performed, 
we also deleted any auxiliary verbs in
the verb phrase. Figure~\ref{figure:rule} contains a sample rule, where verb 
phrases of the type ``was going'' are converted to the verb ``go'', marked as
imperfective.

\begin{figure}[p]
\begin{small}
\begin{alltt}
<pardef n="I">
  <e r="LR"><i><s n="imperf"/></i></e>
  <e r="RL"><i/></e>
</pardef>
\end{alltt}
\end{small}
\caption{Paradigm to select imperfective verbs.}
\label{figure:pardef}
\end{figure}

To facilitate this change, the bilingual dictionary now needed to be
retagged, so that the correct aspect tag was present for selection. We
implemented this as a pair of paradigms (Figure~\ref{figure:pardef}, so that in the English--Polish
direction, the correct verb form is selected, while in the 
Polish--English direction, it is simply discarded. See Figure~\ref{figure:entry}
for a sample entry.

\begin{figure}[p]
\begin{small}
\begin{alltt}
<e>
  <p>
    <l>read<s n="vblex"/></l>
    <r>czyta{\'{c}}<s n="vblex"/></r>
  </p>
  <par n="I"/>
</e>
\end{alltt}
\end{small}
\caption{Sample entry for an imperfective verb, using a paradigm.}
\label{figure:entry}
\end{figure}

\section{Beyond aspect pairs}

As we have mentioned, the situation of Slavic verbs is not quite as
simple as that of just aspect pairs.

In the case of verbs of motion, only a few verbs are affected; however,
as they are also among the most basic, common verbs, we needed to 
address them. 

To select between the determinate and indeterminate forms, we used a
similar system as for aspect, concentrating only on imperfective
phrases. By making the simplistic assumption that the presence or
absence of a preposition immediately after the phrase, we are mostly
successful in choosing the correct form.

When considering frequentative verbs, as aspect is not a problem
(frequentative verbs also have aspect pairs), and the verbs themselves
are falling more and more into decline, we elected simply to lexicalise
the frequentative aspect, so that the verb corresponds with an English
multiword verb containing the word ``often''.\\ 

Dual-aspect verbs are not a particular problem in Polish: \cite{jagod} 
lists only 6; \cite{futrega} has more, but for the most part, they
have been ``adapted'' to modern Polish by the addition of a true
perfective.

Aktionsarten Classes 3 and 4 remained the largest problem.

\section{Aktionsarten}

In addition to a list of verbs from the literature on the 
subject~\citep{aalstein07}, we attempted to extract our own list of verbs
from the Polish Ispell dictionary~\citep{futrega} using the Extract
tool\citep{forsberg07} and a list of verbal prefixes~\citep[5.15.2]{bielec}.
A sample rule is shown in Figure~\ref{figure:extract}.

\begin{figure}[htbp]
\begin{small}
\begin{alltt}
rule class4za = 
x+"ać"
\{ 
  x+"ać" \&
  x+"nąć" \&
  "za"+x+"ać" \&
  ~"za"+x+"ywać" \&
  "po"+x+"ać" \&
  ~"po"+x+"ywać" \&
\}
\end{alltt}
\end{small}
\caption{A sample Extract rule for Class 4 groups where the
empty prefix is {\it za-}.}
\label{figure:extract}
\end{figure}

One class 4 group yielded this way was that of {\it merda\'{c}},
``to wag''. However, in manually checking the forms of this verb,
initially we were only able to find single example of the empty prefix 
perfective, {\it zamerda\'{c}}\footnote{In the sentence ``Ludzie s\k{a}
smutni, bo nie maj\k{a} czym zamerda\'{c}'' -- ``People are sad, because
they have nothing to wag''. Later searches, however, uncovered a much 
more significant number of uses.}, leading us to believe it was not so
important to treat Aktionsarten specially in rules. For some of
the more common verbs which would otherwise have no imperfective
counterpart, we concluded that the best approach was simply to ``invent'' 
a multiword imperfective to match the English multiword, which
proved an adequate solution.

\section{Applicability to Other Languages}

On the whole, we believe that it should be possible to reuse this
work with few changes for other Slavic--English language pairs.

It is, however, important to note that although Slavic languages
basically share the same notions of aspect, some of the details are
quite different, and must be considered even in Slavic--Slavic
translation.

Russian in particular has a great number of dual-aspect verbs, which
should be considered before attempting English--Non-Slavic translation.

We believe that this approach should also be applicable to languages
other than English. For Spanish, for example, it should be much easier
to translate to Slavic, as some of the tenses already include a notion
of aspect; also, there are fewer common analytical verb phrases to
be considered (that is, if a quick view of the verb rules in the
English--Spanish and Spanish--French systems gives an accurate overview).

\section{Future work}
The English--Polish language pair is still a work-in-progress. Though
we have solved the most significant problem in the English--Polish
direction, the current developers' interests lie more in the 
Polish--English side, which still depends on more accurate tagging.

Reports on trigram tagging for Slavic languages in general, and Polish in 
particular, show accuracy of 93--94\%~\citep{HajicKKOP01,Debowski04}; recent
work in Apertium on a trigram tagger~\citep{sheikh2009trigram} is still
ongoing, but looks promising. In addition, some recent translators
have integrated the Constraint Grammar system~\citep{karlsson1990cgf} for
partial disambiguation, which may be able to help with some of the
more difficult ambiguities.

In terms of aspect selection, we are also considering the applicability
of CG to more precise aspect selection: although we have considered
verb phrases, additionally considering other parts of the sentence 
structure, such as subordinating conjunctions, may result in more
accurate translation~\citep{kupsc03,kupsc03a}.

\section{Conclusion}

We have presented a method for aspect selection for machine translators
build with the Apertium Machine Translation platform. We have discussed
an implementation built for the English--Polish language pair, and
discussed the suitability of our approach to other Slavic--Non-Slavic 
pairs.

\section{Acknowledgements}

Thanks to Beata (kt\'{o}ra zainspirowa\l{}a mnie si\k{e} uczy\'{c} polskiego),
whose painstaking word-for-word translation of a manual lead to my 
interest in MT; to Francis Tyers, for his help at the inception of the
English--Polish language pair, for the initial idea at the core of this work,
and for his helpful suggestions for improving this paper; to Sergio Ortiz-Rojas for implementing the
dictionary-less mode that made this possible; to Marcin Mi\l{}kowski for
the vast amount of open-source linguistic resources he has created for Polish;
and to the other members of the Apertium
community, who make it such fun to be involved in.

\bibliographystyle{apalike}
\bibliography{verbs}

\end{document}
